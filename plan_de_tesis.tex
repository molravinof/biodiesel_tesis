\documentclass[a4paper,10pt]{article}
\usepackage[utf8]{inputenc}

%opening
\title{Producción de Biodiesel por Cavitación Hidrodinamica apartir de Aceite usado de cocina}
\author{Juan Miguel Ravelo Jove}

\begin{document}

\maketitle

\begin{abstract}
El agotamiento de los combustibles fosiles, problemas medioambientales, y el encarecimiento del precio del crudo ha incentivado la busqueda de combustibles alternativos. Las caracteristicas del biodiesel ha hecho apuntar a una producción atractiva de biodiesel de alta calidad. El uso de aceite usado de cocina es el componente clave para la reducción del coste de producción a un 60-90\%. Este estudio se enfoca en la producción de biodiesel apartir de aceite usado de cocina, bajo las \emph{implosiones} de la \emph{cavitación hidrodinamica} usando hidroxido de potasio(KOH) como catalisador.
\end{abstract}

\paragraph{Palabras clave:} 
\begin{itemize}
 \item Aceite usado de cocina.
 \item Biodiesel.
 \item Cavitación Hidrodinamica.
 \item Energia de activación.
 \item Transesterificación.
\end{itemize}


\section{Introducción}
Existe varias razones para la busqueda de un combustible alternativo, factibilidad tecnica,
 ambientalmente aceptable, economicamente competitivo y facilmente disponible.
\begin{itemize}
 \item[1.a] La primer principal razón  es el incremento de la demanda de combustible fosiles en todos los sectores de la vida humana$[1]$.
    \begin{itemize}
	  \item transporte
	  \item generación de energia
	  \item procesos industriales
	  \item consumo residencial
    \end{itemize}
 \item[1.b] Esta creciente demanda da lugar a preocupaciones ambientales, tales como grandes 
  emisiones de $CO_2$ y gases de efecto invernadero y tambien el calentamiento global.
 \item[1.c] El consumo mundial de energia se doblo entre 1970 y el 2003
 \item[1.d] El 2006 la \emph{International Energy Outlook} predijo que la demanda mundial de petroleo 
  se incrementara de 84.4 a 116 millones de barriles por dia en USA para el 2030.
 \item[1.e] Actualmente USA consume 97.03 mb/d según la \emph{International Energy Agency}.
	%https://www.iea.org/oilmarketreport/omrpublic/
 \item[2] La segunda razón es que las fuentes de combustibles fosiles no son renovables $[4]$.
 \item[3] La ultima razón es la inestabilidad del precio del crudo$[6]$.
\end{itemize}

Según la \emph{OEFA} el Perú importa 205 mil barriles por dia y produce 63 mil barriles; simplificando el Perú de 3 barriles que consume a diario, 2 los importa y 1 lo produce.
La capacidad del biodiesel de ser usado en lugar del petroleo es una de sus mas importantes caracteristicas(Geyer et al., 1984). \\
Mezclas de 5(BD5) a 20\%(BD20) de biodiesel pueden ser usados en motores diesel existentes sin modificación(Ghorbani et al., 2011).\\
%El biodiesel es un \emph{ester metilico} producido de aceite vegetal o de grasa animal $[8,9]$.
La ASTM(\emph{American Society for Testing and Materials}) define al biodiesel como un ester monoalquilico de cadena larga de acidos grasos, derivados de una fuente renovable lipidica, como aceite vegetal o grasa animal.\\
El mas importante obstaculo en la industrialización y comercialización del biodi- esel es su costo de producción $[33,34]$. Por lo tanto el uso del aceite usado de cocina, reduce su costo de producción en un 60-90\% $[35-39]$. \\
Biodiesel tiene una influencia significativa en la reducción de emisiones del motor tales como hidrocarburos no quemados(68\%), particulas(40\%), monoxido de carbono(44\%), oxido de azufre(100\%) e hidrocarburos aromaticos policiclicos(80-90\%)$[43,44]$

\section{Mecanismos de producción de biodiesel}

\subsection{Pretratamiento del WCO}
La reacción alquilica es muy sensible al contenido de agua y acidos grasos libres $[96,97]$. Una primera filtración eliminará restos solidos, ya filtrado, queda eliminar el agua en un decantador y finalmente nos queda los FFA, el cual se reducira en un filtro de zeolita.

\subsection{Mecanismo de cavitación}
La cavitación es el proceso de nucleación en un liquido cuando la presión cae por debajo de la presión de vapor (Brennen Cavitation and bubble dynamics et al., 1995).
El sistema es compuesto por un tanque reservorio conectado a una bomba centrifuga y a un motor de $1.5\:kW$. La tuberia de descarga fue conectada al tubo principal y  bypass. El tubo principal fue conectado directamente a un plato de orificios para generar la cavitación. Valvulas de estrangulamiento y medidores de presion fueron ajustados para medir la presion.(Cravotto capitulo 3.,2015). En sintesis se espera mediante un estrangulamiento hacer caer la suficiente presión, para generar la nucleación del flujo (metanol/aceite), de modo tal que la implosion que esta genere, en la camara de cavitación, entregue la \emph{energia de activación} necesaria para su reacción.





\end{document}