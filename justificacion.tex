\documentclass[a4paper,10pt]{article}
\usepackage[utf8]{inputenc}

%opening
\title{Biodiesel}
\author{mol ravinof}

\begin{document}

\maketitle

\begin{abstract}
El uso de combustibles alternativos hoy en dia es de uso extensivo, y el reciclaje de aceite usado de cocina, es una opcion para la producción clasica, sólo q se aplicara el HC para producir la transesterificación entregandole la energia de activación suficiente para q esta pueda suceder.
\end{abstract}

\section{Justificación}
Existe varias razones para la busqueda de un combustible alternativo que es tecnicamente factible,
 ambientalmente aceptable, economicamente competitivo y facilmente disponible.
\begin{enumerate}
  \item La primer principal razón  es el incremento de la demanda de combustible fosiles en todos los 
  sectores de la vida humana$[1]$.
  	\begin{itemize}
  		\item transporte
  		\item generación de energia
  		\item procesos industriales
  		\item consumo residencial
  	\end{itemize}
  \item Esta creciente demanda da lugar a preocupaciones ambientales, tales como grandes 
  emisiones de $CO_2$ y gases de efecto invernadero y tambien el calentamiento global.
  \item El consumo mundial de energia se doblo entre 1970 y el 2003
  \item El 2006 la \emph{International Energy Outlook} predijo que la demanda mundial de petroleo 
  se incrementara de 84.4 a 116 millones de barriles por dia en USA para el 2030.
  \item Actualmente USA consume 97.03 mb/d según la \emph{International Energy Agency}.
	%https://www.iea.org/oilmarketreport/omrpublic/
\end{enumerate}
\end{document}
