\documentclass[a4paper,10pt]{article}
\usepackage[utf8]{inputenc}
%opening
\title{Borrador - producción de biodiesel por el método de cavitación hidrodinamica}
\author{mol ravinof}

\begin{document}

\maketitle

\begin{abstract}
El uso de combustibles alternativos hoy en dia es de uso extensivo, y el reciclaje de aceite usado de cocina, es una opcion para la producción clasica, sólo q se aplicara el HC para producir la transesterificación entregandole la energia de activación suficiente para q esta pueda suceder.
\end{abstract}

\section{Parte uno}
\subsection{Aceite}
Es la materia prima a transesterificar, por lo q el aceite reciclado presenta los siguientes elementos:
\begin{itemize}
 \item Carbono quemado.
 \item Radicales libres.
 \item Agua.
 \item Restos organicos.
\end{itemize}

Y en respuesta a esto, se tiene q ver el modo de eliminar estas interferencias para la reacción de transesterificación.
\begin{enumerate}
 \item Filtración simple a malla 50. (-restos organicos)
 \item Decantación. (-agua)
 \item Filtración con arena y zeolita. (-C y radicales)
\end{enumerate}

\subsection{Alcohol}
Parte de los reactantes, viene al 99\%, por q no hay mucho lio con esto.

\subsection{Catalisador}
El KOH, permite elevar la energia de activación, reaccionando este con el meOH formando me-O-K, el cual por ser mas reactivo q el simple meOH, produce la transesterificación.

\subsection{HC}
Al igual que el Catalisador, su función de la cavitación es entregarle Ea suficiente a los reactantes de modo q se rompan las cadenas del triglicerido.
\begin{itemize}
 \item cavitación mediante reducción y apertura (venturi).
 \item cavitación mediante una valvula.
 \item cavitación mediante un rotor.
 \item cavitación mediante una elice(barcos).
 \item cavitación mediante ultrasonido.
\end{itemize}

lo que se busca a priori es evitar q tal fenomeno ocurra dentro de la bomba y esta ocurra en una camara de cavitación, (\textquotedblleft \emph{cavitation chamber} \textquotedblright)

\section{Parte dos}
\subsection{Biodiesel}
Tal producto, debe ser filtrado con el fin de remover, la glicerina como subproducto de la transesterificación. No habra cambio alguno del modelo de planta conocido, con aserrin. aunq puede ser contrastado con la bateria de columnas de zeolita.

\subsection{Glicerina}
Se buscara purificar.

\section{Parte tres}
\subsection{variables de control}
\begin{itemize}
 \item temperatura
 \item pH
 \item entalpia
 \item indice de cetano
 \item indice de yodo
 \item punto flash
\end{itemize}


\end{document}