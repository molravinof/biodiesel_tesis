\documentclass[a4paper,10pt]{article}
\usepackage[utf8]{inputenc}

%opening
\title{Marco teorico del biodiesel}
\author{Mol Ravinof}

\begin{document}

\maketitle

\begin{abstract}
Para que acepten mi tema de tesis, primero debo presentar mi plan de tesis; las definiciones del proceso a realizar, palabras claves y esas cosas, un modelo basico de ecuaciones diferenciales, objetivos, q prentendo obtener, y para que, etc, etc.
\end{abstract}

\section{Identificación de material usado}
\paragraph{Material pretratamiento al aceite}
\begin{itemize}
 \item Cilindro 15, 60 gln
 \item Criba
 \item Malla 150
 \item Tanque decantador de 100 gln
 \item Baterial de filtros (tuberias)
 \item arena
 \item zeolita
\end{itemize}


\paragraph{Material de la Reacción}
\begin{itemize}
 \item Biodiesel
 \item Glicerina
 \item meOH
 \item KOH
 \item Waste cooking oil(WCO)
\end{itemize}
\paragraph{Material de Reacción-equipos}
\begin{itemize}
 \item Tanque Reactor-Mezclador de 50 gln
 \item Tanque del meOH-KOH
 \item Intercambiador de calor
 \item centrifugal pump
 \item bomba hidraulica
 \item camara de Cavitación
 \item tubos venturi
 \item ultrasonic transducer
 \item Tanque deposito, decantador (100 gln)
\end{itemize}

\paragraph{Zona de filtrado del Biodiesel}
\begin{itemize}
 \item Tanque deposito, biodiesel sucio
 \item Bomba hidraulica
 \item Tanque filtro de azerrin
 \item Bateria de filtro en $\mu$m
 \item Tanque deposito biodiesel limpio
\end{itemize}

\section{Fenomenos que se presentaran}
  \subsection{Transporte de masa}
    Esto ocurre en la area de pretratamiento, filtrando y decantando
  \subsection{Transporte de energia}
    La energia suministrada por la cavitacion es para aumentar la energia de activación y llevar a cabo la reacción
    \subsubsection{Recuperación de meOH}
      En este caso es para recuperar destilando de modo alguno el metanol contenido en el biodiesel y en la reacción del metoxido
\section{Operaciones unitarias a realizar}
  \subsection{Filtración}
  Busca eliminar material organico, Carbon, acidos grasos libres
  \subsubsection{Analisis}
    Se hara el analisis de calidad del WCO con el fin de evaluar la efectividad del filtrado.\\
    En especial para la determinación de acidos grasos libres.
  \subsection{Mezclado}
  El mezclado nos dara los ratios a evaluar, las proporciones oil/meOH/KOH
  \subsection{Decantación}
  Busca eliminar el agua contenido en el WCO
\section{Procesos unitarios a realizar}
  \subsection{Transesterificación}
  Principal reacción para la obtención de biodiesel
  \subsection{Reacción del metoxido}
  Reacción del catalizador con el meOH
  \subsection{Cavitación hidrodinamica}
  Medio por nucleación e implosion que otorga Ea suficiente para la transesterificación
  \subsection{Sonochemical Reaction}
  Medio acustico que entre longitudes de onda a la Ea sufuciente para la transesterificación
\end{document}